% Politecnico di Milano (PoliMi) - School of Industrial and Information Engineering
%
% Last Revision: October 2021
%
% Copyright 2021 Politecnico di Milano, Italy. NC-BY

\documentclass{Configuration_Files/Template}

%------------------------------------------------------------------------------
%	REQUIRED PACKAGES AND  CONFIGURATIONS
%------------------------------------------------------------------------------

% CONFIGURATIONS
\usepackage{parskip} % For paragraph layout
\usepackage{setspace} % For using single or double spacing
\usepackage{emptypage} % To insert empty pages
\usepackage{multicol} % To write in multiple columns (executive summary)
\setlength\columnsep{15pt} % Column separation in executive summary
\setlength\parindent{0pt} % Indentation
\raggedbottom  

% PACKAGES FOR TITLES
\usepackage{titlesec}
% \titlespacing{\section}{left spacing}{before spacing}{after spacing}
\titlespacing{\section}{0pt}{3.3ex}{2ex}
\titlespacing{\subsection}{0pt}{3.3ex}{1.65ex}
\titlespacing{\subsubsection}{0pt}{3.3ex}{1ex}
\usepackage{color}

% PACKAGES FOR LANGUAGE AND FONT
\usepackage[english]{babel} % The document is in English  
\usepackage[utf8]{inputenc} % UTF8 encoding
\usepackage[T1]{fontenc} % Font encoding
\usepackage[11pt]{moresize} % Big fonts

% PACKAGES FOR IMAGES
\usepackage{graphicx}
\usepackage{transparent} % Enables transparent images
\usepackage{eso-pic} % For the background picture on the title page
\usepackage{subfig} % Numbered and caption subfigures using \subfloat.
\usepackage{tikz} % A package for high-quality hand-made figures.
\usetikzlibrary{}
\graphicspath{{./Images/}} % Directory of the images
\usepackage{caption} % Coloured captions
\usepackage{amsthm,thmtools,xcolor} % Coloured "Theorem"
\usepackage{float}

% STANDARD MATH PACKAGES
\usepackage{amsmath}
\usepackage{amsthm}
\usepackage{amssymb}
\usepackage{amsfonts}
\usepackage{bm}
\usepackage[overload]{empheq} % For braced-style systems of equations.
\usepackage{fix-cm} % To override original LaTeX restrictions on sizes

% PACKAGES FOR TABLES
\usepackage{tabularx}
\usepackage{longtable} % Tables that can span several pages
\usepackage{colortbl}

% PACKAGES FOR ALGORITHMS (PSEUDO-CODE)
\usepackage{algorithm}
\usepackage{algorithmic}

% PACKAGES FOR REFERENCES & BIBLIOGRAPHY
\usepackage[colorlinks=true,linkcolor=black,anchorcolor=black,citecolor=black,filecolor=black,menucolor=black,runcolor=black,urlcolor=black]{hyperref} % Adds clickable links at references
\usepackage{cleveref}
\usepackage[square, numbers, sort&compress]{natbib} % Square brackets, citing references with numbers, citations sorted by appearance in the text and compressed
\bibliographystyle{abbrvnat} % You may use a different style adapted to your field

% OTHER PACKAGES
\usepackage{pdfpages} % To include a pdf file
\usepackage{afterpage}
\usepackage{lipsum} % DUMMY PACKAGE
\usepackage{fancyhdr} % For the headers
\fancyhf{}

% Input of configuration file. Do not change config.tex file unless you really know what you are doing. 
% Define blue color typical of polimi
\definecolor{bluepoli}{cmyk}{0.4,0.1,0,0.4}

% Custom theorem environments
\declaretheoremstyle[
  headfont=\color{bluepoli}\normalfont\bfseries,
  bodyfont=\color{black}\normalfont\itshape,
]{colored}

% Set-up caption colors
\captionsetup[figure]{labelfont={color=bluepoli}} % Set colour of the captions
\captionsetup[table]{labelfont={color=bluepoli}} % Set colour of the captions
\captionsetup[algorithm]{labelfont={color=bluepoli}} % Set colour of the captions

\theoremstyle{colored}
\newtheorem{theorem}{Theorem}[chapter]
\newtheorem{proposition}{Proposition}[chapter]

% Enhances the features of the standard "table" and "tabular" environments.
\newcommand\T{\rule{0pt}{2.6ex}}
\newcommand\B{\rule[-1.2ex]{0pt}{0pt}}

% Pseudo-code algorithm descriptions.
\newcounter{algsubstate}
\renewcommand{\thealgsubstate}{\alph{algsubstate}}
\newenvironment{algsubstates}
  {\setcounter{algsubstate}{0}%
   \renewcommand{\STATE}{%
     \stepcounter{algsubstate}%
     \Statex {\small\thealgsubstate:}\space}}
  {}

% New font size
\newcommand\numfontsize{\@setfontsize\Huge{200}{60}}

% Title format: chapter
\titleformat{\chapter}[hang]{
\fontsize{50}{20}\selectfont\bfseries\filright}{\textcolor{bluepoli} \thechapter\hsp\hspace{2mm}\textcolor{bluepoli}{|   }\hsp}{0pt}{\huge\bfseries \textcolor{bluepoli}
}

% Title format: section
\titleformat{\section}
{\color{bluepoli}\normalfont\Large\bfseries}
{\color{bluepoli}\thesection.}{1em}{}

% Title format: subsection
\titleformat{\subsection}
{\color{bluepoli}\normalfont\large\bfseries}
{\color{bluepoli}\thesubsection.}{1em}{}

% Title format: subsubsection
\titleformat{\subsubsection}
{\color{bluepoli}\normalfont\large\bfseries}
{\color{bluepoli}\thesubsubsection.}{1em}{}

% Shortening for setting no horizontal-spacing
\newcommand{\hsp}{\hspace{0pt}}

\makeatletter
% Renewcommand: cleardoublepage including the background pic
\renewcommand*\cleardoublepage{%
  \clearpage\if@twoside\ifodd\c@page\else
  \null
  \AddToShipoutPicture*{\BackgroundPic}
  \thispagestyle{empty}%
  \newpage
  \if@twocolumn\hbox{}\newpage\fi\fi\fi}
\makeatother

%For correctly numbering algorithms
\numberwithin{algorithm}{chapter}

%----------------------------------------------------------------------------
%	NEW COMMANDS DEFINED
%----------------------------------------------------------------------------

% EXAMPLES OF NEW COMMANDS
\newcommand{\bea}{\begin{eqnarray}} % Shortcut for equation arrays
\newcommand{\eea}{\end{eqnarray}}
\newcommand{\e}[1]{\times 10^{#1}}  % Powers of 10 notation

%----------------------------------------------------------------------------
%	ADD YOUR PACKAGES (be careful of package interaction)
%----------------------------------------------------------------------------

\usepackage{geometry}
\usepackage{tabularx}
\usepackage{booktabs,xltabular}
\usepackage{hyperref}
\usepackage{listings}

%----------------------------------------------------------------------------
%	ADD YOUR DEFINITIONS AND COMMANDS (be careful of existing commands)
%----------------------------------------------------------------------------

% Set uniform margins
\geometry{
  left=0.8in,
  right=0.8in,
  top=1in,
  bottom=1in,
  includehead,
  includefoot
}

%----------------------------------------------------------------------------
%	BEGIN OF YOUR DOCUMENT
%----------------------------------------------------------------------------

\begin{document}

\fancypagestyle{plain}{%
\fancyhf{} % Clear all header and footer fields
\fancyhead[RO,RE]{\thepage} %RO=right odd, RE=right even
\renewcommand{\headrulewidth}{0pt}
\renewcommand{\footrulewidth}{0pt}}

%----------------------------------------------------------------------------
%	TITLE PAGE
%----------------------------------------------------------------------------

\pagestyle{empty} % No page numbers
\frontmatter % Use roman page numbering style (i, ii, iii, iv...) for the preamble pages

\puttitle{
    title= ATD Document,
    name= {Mattia Piccinato, Gabriele Puglisi, Jacopo Piazzalunga},
    academicyear= {2023-24},
    link= \href{https://github.com/GabP404/PiazzalungaPiccinatoPuglisi-CodeKataBattle}{Click here}

 }

%----------------------------------------------------------------------------
%	PREAMBLE PAGES: ABSTRACT (inglese e italiano), EXECUTIVE SUMMARY
%----------------------------------------------------------------------------
\startpreamble
\setcounter{page}{1} % Set page counter to 1

%----------------------------------------------------------------------------
%	LIST OF CONTENTS/FIGURES/TABLES/SYMBOLS
%----------------------------------------------------------------------------

% TABLE OF CONTENTS
\thispagestyle{empty}
\tableofcontents % Table of contents 
\thispagestyle{empty}
\cleardoublepage

%-------------------------------------------------------------------------
%	MAIN TEXT
%-------------------------------------------------------------------------.

\addtocontents{toc}{\vspace{2em}} % Add a gap in the Contents, for aesthetics
\mainmatter % Begin numeric (1,2,3...) page numbering


% FIRST CHAPTER
% --------------------------------------------------------------------------
\chapter{Tested Project}

\begin{itemize}

    \item \textbf{Authors:}
    
    Polito Attilio\\
    Rigione Pisone Raimondo\\
    Soricelli Francesco\\
    
    \item \textbf{Github Repository:}
    
    \href{https://github.com/Attilioap/PolitoRigionePisoneSoricelli}{Click here to see the repository}\\
    
    \item \textbf{Reference Documents:}\\
    \href{https://github.com/Attilioap/PolitoRigionePisoneSoricelli/blob/main/DeliveryFolder/RASDV2.pdf}{RASD: Click here}\\
    \href{https://github.com/Attilioap/PolitoRigionePisoneSoricelli/blob/main/DeliveryFolder/DDV2.pdf}{DD: Click here}\\
    \href{https://github.com/Attilioap/PolitoRigionePisoneSoricelli/blob/main/DeliveryFolder/ITD1.pdf}{ITD: Click here}\\
    
\end{itemize}

% SECOND CHAPTER
% --------------------------------------------------------------------------
\chapter{Installation}

\begin{itemize}

    \item \textbf{1.} We utilized a Windows Sandbox as our testing environment.
    \item \textbf{2.} We cloned the repository.
    \item \textbf{3.} We moved the folder "IT/Implementazione" to the desktop.
    \item \textbf{4.} We installed Python version 3.10.0 on the machine.
    \item \textbf{5.} We added the path to python.exe to the PATH environment variable.
    \item \textbf{6.} We opened the Command Prompt.
    \item \textbf{7.} We navigated to the "Implementazione" folder on the Desktop.
    \item \textbf{8.} We executed the following commands:
    
    \begin{lstlisting}
    python -m venv env
    env\Scripts\activate
    pip install django
    pip install django-crispy-forms
    pip install github
    pip install PyGithub
    pip install flake8
    cd codekatabattle
    python manage.py runserver
    \end{lstlisting}
    
    \item \textbf{9.} We accessed the login interface at http://127.0.0.1:8000/ckbapp/login.
    \item \textbf{10.} We registered on ngrok.com.
    \item \textbf{11.} We downloaded the Windows x64 zip folder.
    \item \textbf{12.} We extracted the contents of the zip folder.
    \item \textbf{13.} We opened the ngrok.exe terminal located within the zip folder.
    \item \textbf{14.} Using the token provided immediately after clicking the download button, we executed the following commands:
    
    \begin{lstlisting}
    ngrok config add-authtoken <token>
    ngrok http http://localhost:8000
    \end{lstlisting}
    
    \item \textbf{15.} We copied the public endpoint, navigated to "IT/Implementazione/codekatabattle/codekatabattle", and replaced the endpoint URL in the settings.py file as specified in the installation guide.
    
\end{itemize}

The installation process proceeded smoothly without any issues.

{\color{bluepoli}\rule{\linewidth}{0.1pt}}

% THIRD CHAPTER
% --------------------------------------------------------------------------
\chapter{Test cases}

\section{Premise}

It is our responsibility to specify that, upon starting the server locally, some issues arose with the display of the contents, resulting in a subsequent loss in terms of usability.

Specifically, the pages would repeatedly redirect to themselves until manually interrupted. Additionally, upon clicking, the browser would display the page's HTML without CSS.

However, during testing, we observed that after sufficient navigation, the server stabilized and began displaying the pages correctly.

Thus, we were able to conduct testing without encountering any other issues.

This premise merely serves to highlight such undesirable behavior, which warrants greater attention.

\section{Testing}

This section will outline the testing procedures we conducted to evaluate both functionality and adherence to the project's product functions as outlined in the provided RASD.

Each test case was designed to encompass the most relevant usage scenarios from the User's perspective.

Below is a comprehensive list of the test cases executed to validate the product's performance and compliance.

For each, we have stated the goal, the steps performed, the cases considered, and the results obtained.\\

\begin{itemize}

    \item \textbf{F1 - Registration and Login}
    
    \begin{enumerate}
    
        \item Goal:
        
        Verify that the registration and login processes work as expected, adequately addressing security measures while ensuring a seamless user experience in terms of usability and accessibility.
        
        \item Steps:
        
            1. Create a Student according to the Domain Assumption D10\\
            2. Login\\
            3. Logout\\
            4. Create an Educator\\
            5. Login\\
            6. Logout
        
        \item Test cases:
        
        \begin{enumerate}
        
            \item Correct flow
            \item Already existing username
            \item Username case-sensitivity
            \item Invalid email field
            \item Weak trivial password
    
        \end{enumerate}
        
        \item Results:
        
        The test results indicate that the registration and login processes of the app generally proceed smoothly, exhibiting user-friendly and secure functionalities.
        
        However, there is a lack of enforcement regarding password strength, as weak passwords such as "a" were accepted during registration.\\
    
    \end{enumerate}
    
    \item \textbf{F2 - Creation, Management and Closure of Tournaments}
    
    \begin{enumerate}
    
        \item Goal:
        
        Verify that the Educators are indeed able to create and close a Tournament, and creating Battles in the context of a Tournament.
        
        \item Steps:
        
            1. Login as an Educator\\
            2. Create a Tournament\\
            3. Create a Battle for the Tournament\\
            4. Log out\\
            4. Access the database with administration privilege\\
            5. Move up the subscription deadline of the Tournament\\
            6. Move up the subscription deadline of the Battle\\
            7. Go back to the login interface\\
            8. Log in as a Student\\
            9. Visualize the Battle repository link in the Battle's status page\\
            10. Logout\\
            11. Login as the Educator who created the Tournament\\
            12. Close the Tournament 
        
        \item Test cases:
        
        \begin{enumerate}
        
            \item Correct flow
            \item Invalid maximum number of students
            \item Any deadline in the past
            \item Subscription deadline after the submission deadline
            \item Submission deadline after the Tournament submission deadline
            \item Wrong file format
    
        \end{enumerate}
        
        \item Results:
        
        The test results suggest that the process of creating, managing, and closing tournaments generally runs smoothly, with secure functionalities in place.

        However, there's no validation on the file format, which is expected to be a Python file according to Chapter 5 in the ITD. This oversight could lead to problems when creating the GitHub repository for the Battle.

        Furthermore, when invalid parameters are provided in the request, the error visualization on the client-side isn't always clear. For instance, when a negative maximum number of students is specified.\\
    
    \end{enumerate}
    
    \item \textbf{F3 - Delegation of Tournament Management}
    
    \begin{enumerate}
    
        \item Goal:
        
        Verify that an Educator who manages a Tournament can delegate the management of such Tournament to another Educator.
        
        \item Steps:
        
            1. Login as Educator A\\
            2. Create a Tournament\\
            3. Delegate Educator B\\
            4. Logout\\
            5. Login as Educator B\\
            6. Create a Battle in the Tournament\\
            7. Delegate Educator C\\
            8. Logout\\
            9. Login as Educator C\\
            10. Create a Battle in the Tournament\\
            11. Access the database with administration privilege\\
            12. Move up the subscription deadline of the Tournament\\
            13. Go back to the Educator interface\\
            14. Close the Tournament 
        
        \item Test cases:
        
        \begin{enumerate}
        
            \item Correct workflow
            \item Delegating the Educator itself
            \item Delegating an Educator who already manages the Tournament
            \item Delegating a Student
            \item Delegating a User who does not exist
    
        \end{enumerate}
        
        \item Results:
        
        The test results confirm that Educators can indeed delegate the management of a Tournament they oversee to other Educators, but they cannot delegate it to Students.
        
        Additionally, the User is appropriately notified if they attempt to delegate an Educator who already has management privileges, ensuring clarity and preventing inadvertent actions that may lead to confusion or errors in the system.

        However, it also results that Educators who are allowed by the creator to manage a Tournament can only create Battles, as they cannot access the Tournament status page, making it impossible for them to close the Tournament.\\
    
    \end{enumerate}
    
    \item \textbf{F4 - Exploration of Coding Challenges}
    
    \begin{enumerate}
    
        \item Goal:
        
        Verify that a Student can view all available Tournaments and Battles for subscription, ensuring they have access only to those that are relevant to their eligibility or criteria.
        
        \item Steps:
        
            1. Login as Educator\\
            2. Create a Tournament 1\\
            3. Create a Tournament 2\\
            4. Create a Battle in Tournament 2\\
            5. Logout\\
            6. Access the database with administration privilege\\
            7. Move up the subscription deadline of Tournament 1\\
            8. Go back to the login interface\\
            9. Login as Student\\
            10. Visualize available Tournaments\\
            11. Enroll in Tournament 2\\
            12. Visualize available Battles in Tournament 2\\
            13. Access the database with administration privilege\\
            14. Move up the submission deadline of Tournament 2\\
            15. Go back to the Student interface\\
            16. Visualize available Tournaments
        
        \item Test cases:
        
        \begin{enumerate}
        
            \item Correct flow
    
        \end{enumerate}
        
        \item Results:
        
        The test results indicate that students can visualize all the information they should be able to.
        
        Additionally, they do not see expired Tournaments and Battles that they can no longer enroll in. Furthermore, they continue to see Battles that they have participated in, maintaining a consistent and relevant user experience.\\
    
    \end{enumerate}
    
    \item \textbf{F5 - Participation in Coding Battles}
    
    \begin{enumerate}
    
        \item Goal:
        
        Verify that Students can take part in Battles and Tournaments which are in their subscription phase.
        
        \item Steps:
        
            1. Login as Educator\\
            2. Create a Tournament\\
            3. Create a Battle in the Tournament\\
            4. Logout\\
            5. Login as Student\\
            6. Enroll in the Tournament\\
            7. Enroll in the Battle
        
        \item Test cases:
        
        \begin{enumerate}
        
            \item Correct flow
            \item Invalid Tournament id
            \item Invalid Battle id
    
        \end{enumerate}
        
        \item Results:
        
        The test results indicate that users can effectively and exclusively enroll in tournaments and battles that they should be allowed to participate in.
        
        However, we advise the development team to always refresh the page and display an error message to the User in case of a bad request, in order to provide clear guidance and enhance the User experience.\\
    
    \end{enumerate}

    \item \textbf{F6 - Team Formation} 
    
    \begin{enumerate}
    
        \item Goal: 
        
        Verify that Students who have enrolled in a Battle are allowed to form a team by inviting other Students, up to the maximum specified limit.
        
        \item Steps:
        
            1. Login as Educator\\
            2. Create a Tournament\\
            3. Create a Battle in the Tournament with maximum 3 Students per team\\
            4. Logout\\
            5. Login as Student A\\
            6. Enroll in the Tournament\\
            7. Enroll in the Battle\\
            8. Invite 2 Students B and C\\
            9. Logout\\
            10. Login as Student B\\
            11. Enroll in the Tournament\\
            12. Enroll in the Battle\\
            13. Accept the invite\\
            14. Logout\\
            15. Login as Student B\\
            16. Enroll in the Tournament\\
            17. Enroll in the Battle\\
            18. Accept the invite
        
        \item Test cases:
        
        \begin{enumerate}
        
            \item Correct flow
            \item Invite more Students than allowed
            \item Invite a Student as an invited Student
            \item Invite a Student who is already in a team for the Battle
            \item Invite a Student who is not enrolled in the Tournament
    
        \end{enumerate}
        
        \item Results:
        
        The test results confirm that students who participate in a battle can successfully and exclusively form teams comprising up to the maximum number of students. These students must be enrolled in the tournament and not yet enrolled in the battle.

        However, the frontend appears to display a "useless" search bar to Students who have joined a Battle by invitation, despite the system's restriction on their ability to send invitations. It would be more beneficial if this search bar were not presented, thereby eliminating confusion and streamlining the User experience.\\
    
    \end{enumerate}
    
    \item \textbf{F7 - Real-time Competition and Scores}
    
    \begin{enumerate}
    
        \item Goal:
        
        Verify that results of each commit for a Battle are updated in real-time.
        
        \item Steps:
        
            1. Login as an Educator\\
            2. Create a Tournament\\
            3. Create a Battle\\
            4. Logout\\
            5. Login as a Student\\
            6. Enroll in the Tournament\\
            7. Enroll in the Battle\\
            8. Access the database with administration privilege\\
            9. Move up the subscription deadline of the Tournament\\
            10. Move up the subscription deadline of the Battle\\
            11. Go back to the Student interface\\
            12. Click on the link of the repository\\
            13. Sign up to GitHub according to the Domain Assumption D10\\
            14. Fork the repository\\
            15. Create the yaml.yml file according to the installation guide and commit\\
            16. Edit the code\_kata.py file in the folder code\_katas folder of the repository and commit\\
            17. Go back to the Student interface
            18. Refresh the page
        
        \item Test cases:
        
        \begin{enumerate}
        
            \item Correct flow
            \item Student who is enrolled in the Battle in a team
            \item Student who is not enrolled in the Battle
    
        \end{enumerate}
        
        \item Results:
        
        The test results indicate that the system successfully updates the rankings of both Battles and Tournaments when triggered by a push from a Student who is indeed participating in the Battle. Additionally, it is designed to ignore commits made by Students who do not have the appropriate permissions.\\
    
    \end{enumerate}

    \item \textbf{F8 - Manual Evaluation}
    
    \begin{enumerate}
    
        \item Goal:
        
        Verify that Educators can perform a Manual Evaluation at the end of the Battle.
        
        \item Steps:
        
            1. Login as an Educator\\
            2. Create a Tournament\\
            3. Create a Battle\\
            4. Logout\\
            5. Login as Student A\\
            6. Enroll in the Tournament\\
            7. Enroll in the Battle\\
            8. Logout\\
            9. Login as Student B\\
            10. Enroll in the Tournament\\
            11. Enroll in the Battle\\
            12. Logout\\
            13. Access the database with administration privilege\\
            14. Move up the subscription deadline of the Tournament\\
            15. Move up the subscription deadline of the Battle\\
            16. Create GitHub accounts for the two Students according to D10\\
            17. Go back to the login interface\\
            18. Login as Student A\\
            19. Click on the link of the repository\\
            20. Fork the repository\\
            21. Create the yaml.yml file according to the installation guide and commit\\
            22. Edit the code\_kata.py file in the folder code\_katas folder of the repository and commit\\
            23. Go back to the Student interface\\
            24. Logout\\
            25. Repeat everything from step 18 to step 24 with Student B\\
            26. Access the database with administration privilege\\
            27. Move up the submission deadline of the Battle\\
            28. Login as the Educator who created the Tournament\\
            29. Perform Manual Evaluation
        
        \item Test cases:
        
        \begin{enumerate}
        
            \item Correct flow
            \item Close Tournament while Battle is ongoing
             
        \end{enumerate}
        
        \item Results:
        
        The test results indicate that the Educators are indeed always able to perform Manual Evaluation, for every team who participated in the Battle, regardless how the Battle was stopped.\\
    
    \end{enumerate}

    \item \textbf{F9 - Results Display}
    
    \begin{enumerate}
    
        \item Goal:
        
        Verify that results are promptly displayed upon the conclusion of a Tournament or Battle, and that they are coherent and comprehensive, considering all commits made before the deadline.
        
        \item Steps:
        
            1. Log in as an Educator\\
            2. Create a Tournament\\
            3. Create a Battle\\
            4. Log out\\
            5. Log in as Student A\\
            6. Enroll in the Tournament\\
            7. Enroll in the Battle\\
            8. Log out\\
            9. Log in as Student B\\
            10. Enroll in the Tournament\\
            11. Enroll in the Battle\\
            12. Log out\\
            13. Access the database with administrative privileges\\
            14. Move up the subscription deadline of the Tournament\\
            15. Move up the subscription deadline of the Battle\\
            16. Create GitHub accounts for the two Students according to D10\\
            17. Return to the login interface\\
            18. Log in as Student A\\
            19. Click on the link of the repository\\
            20. Fork the repository\\
            21. Create the yaml.yml file according to the installation guide and commit\\
            22. Edit the code\_kata.py file in the code\_katas folder of the repository and commit\\
            23. Return to the Student interface\\
            24. Log out\\
            25. Repeat steps 18 to 24 with Student B\\
            26. Access the database with administrative privileges\\
            27. Move up the submission deadline of the Battle\\
            28. Move up the submission deadline of the Tournament\\
            29. Return to the login interface\\
            30. Log in as Student A\\
            31. Visualize the results\\
            32. Log out\\
            33. Log in as the Educator\\
            34. Perform Manual Evaluation\\
            35. Log out\\
            36. Log in as Student A\\
            37. Visualize the results\\
        
        \item Test cases:
        
        \begin{enumerate}
        
            \item Correct flow
    
        \end{enumerate}
        
        \item Results:
        
        The test results indicate that results are correctly displayed, both before and after Manual Evaluation by the Educators managing the Tournament.\\
    
    \end{enumerate}

    
    \item \textbf{F10 - Timely Notification}
    
    According to the ITD provided to us, the Timely Notification product function was not implemented.\\

    \item \textbf{F11 - Detailed Student Profile}

    Since gamification aspects were not among the required functions to be developed, we did not verify whether the student profiles were implemented, as their sole purpose was to display the Badges.\\

\end{itemize}

\section{Further notes}

At the conclusion of the entire testing process, it's important to note the system's behavior regarding HTTP status codes.

We observed instances where the system returned inappropriate codes, such as frequently issuing a status code of 200 even when the request wasn't adequately fulfilled.

We advise the development team to ensure adherence to the HTTP communication standard to improve interoperability, compatibility, and integration with third-party services. Aligning with this standard would indeed result in a more robust and user-friendly experience for all stakeholders involved.

Nevertheless, the prototype provided to us is fully functional and appears to implement all the intended functions correctly, securely, and in a user-friendly manner.

{\color{bluepoli}\rule{\linewidth}{0.1pt}}

% FOURTH CHAPTER
% --------------------------------------------------------------------------
\chapter{Effort spent}

This section shows the amount of time that each member has spent to produce the document. Please notice that each unit is the result of coordinated work among all the members.

\begin{table}[h]
\centering
\begin{tabularx}{\textwidth}{| X | X | X |}
\hline
\textbf{UNIT} & \textbf{MEMBERS} & \textbf{HOURS} \\ [1ex]
\hline
Testing & Piccinato, Piazzalunga, Puglisi & 20h \\ [1ex]
\hline
Report & Piccinato & 10h \\ [1ex]
\hline
\end{tabularx}
\end{table}

{\color{bluepoli}\rule{\linewidth}{0.1pt}}

\end{document}
