% A LaTeX template for MSc Thesis submissions to 
% Politecnico di Milano (PoliMi) - School of Industrial and Information Engineering
%
% S. Bonetti, A. Gruttadauria, G. Mescolini, A. Zingaro
% e-mail: template-tesi-ingind@polimi.it
%
% Last Revision: October 2021
%
% Copyright 2021 Politecnico di Milano, Italy. NC-BY

\documentclass{Configuration_Files/PoliMi3i_thesis}

%------------------------------------------------------------------------------
%	REQUIRED PACKAGES AND  CONFIGURATIONS
%------------------------------------------------------------------------------

% CONFIGURATIONS
\usepackage{parskip} % For paragraph layout
\usepackage{setspace} % For using single or double spacing
\usepackage{emptypage} % To insert empty pages
\usepackage{multicol} % To write in multiple columns (executive summary)
\setlength\columnsep{15pt} % Column separation in executive summary
\setlength\parindent{0pt} % Indentation
\raggedbottom  

% PACKAGES FOR TITLES
\usepackage{titlesec}
% \titlespacing{\section}{left spacing}{before spacing}{after spacing}
\titlespacing{\section}{0pt}{3.3ex}{2ex}
\titlespacing{\subsection}{0pt}{3.3ex}{1.65ex}
\titlespacing{\subsubsection}{0pt}{3.3ex}{1ex}
\usepackage{color}

% PACKAGES FOR LANGUAGE AND FONT
\usepackage[english]{babel} % The document is in English  
\usepackage[utf8]{inputenc} % UTF8 encoding
\usepackage[T1]{fontenc} % Font encoding
\usepackage[11pt]{moresize} % Big fonts

% PACKAGES FOR IMAGES
\usepackage{graphicx}
\usepackage{transparent} % Enables transparent images
\usepackage{eso-pic} % For the background picture on the title page
\usepackage{subfig} % Numbered and caption subfigures using \subfloat.
\usepackage{tikz} % A package for high-quality hand-made figures.
\usetikzlibrary{}
\graphicspath{{./Images/}} % Directory of the images
\usepackage{caption} % Coloured captions
\usepackage{xcolor} % Coloured captions
\usepackage{amsthm,thmtools,xcolor} % Coloured "Theorem"
\usepackage{float}

% STANDARD MATH PACKAGES
\usepackage{amsmath}
\usepackage{amsthm}
\usepackage{amssymb}
\usepackage{amsfonts}
\usepackage{bm}
\usepackage[overload]{empheq} % For braced-style systems of equations.
\usepackage{fix-cm} % To override original LaTeX restrictions on sizes

% PACKAGES FOR TABLES
\usepackage{tabularx}
\usepackage{longtable} % Tables that can span several pages
\usepackage{colortbl}

% PACKAGES FOR ALGORITHMS (PSEUDO-CODE)
\usepackage{algorithm}
\usepackage{algorithmic}

% PACKAGES FOR REFERENCES & BIBLIOGRAPHY
\usepackage[colorlinks=true,linkcolor=black,anchorcolor=black,citecolor=black,filecolor=black,menucolor=black,runcolor=black,urlcolor=black]{hyperref} % Adds clickable links at references
\usepackage{cleveref}
\usepackage[square, numbers, sort&compress]{natbib} % Square brackets, citing references with numbers, citations sorted by appearance in the text and compressed
\bibliographystyle{abbrvnat} % You may use a different style adapted to your field

% OTHER PACKAGES
\usepackage{pdfpages} % To include a pdf file
\usepackage{afterpage}
\usepackage{lipsum} % DUMMY PACKAGE
\usepackage{fancyhdr} % For the headers
\fancyhf{}

% Input of configuration file. Do not change config.tex file unless you really know what you are doing. 
% Define blue color typical of polimi
\definecolor{bluepoli}{cmyk}{0.4,0.1,0,0.4}

% Custom theorem environments
\declaretheoremstyle[
  headfont=\color{bluepoli}\normalfont\bfseries,
  bodyfont=\color{black}\normalfont\itshape,
]{colored}

% Set-up caption colors
\captionsetup[figure]{labelfont={color=bluepoli}} % Set colour of the captions
\captionsetup[table]{labelfont={color=bluepoli}} % Set colour of the captions
\captionsetup[algorithm]{labelfont={color=bluepoli}} % Set colour of the captions

\theoremstyle{colored}
\newtheorem{theorem}{Theorem}[chapter]
\newtheorem{proposition}{Proposition}[chapter]

% Enhances the features of the standard "table" and "tabular" environments.
\newcommand\T{\rule{0pt}{2.6ex}}
\newcommand\B{\rule[-1.2ex]{0pt}{0pt}}

% Pseudo-code algorithm descriptions.
\newcounter{algsubstate}
\renewcommand{\thealgsubstate}{\alph{algsubstate}}
\newenvironment{algsubstates}
  {\setcounter{algsubstate}{0}%
   \renewcommand{\STATE}{%
     \stepcounter{algsubstate}%
     \Statex {\small\thealgsubstate:}\space}}
  {}

% New font size
\newcommand\numfontsize{\@setfontsize\Huge{200}{60}}

% Title format: chapter
\titleformat{\chapter}[hang]{
\fontsize{50}{20}\selectfont\bfseries\filright}{\textcolor{bluepoli} \thechapter\hsp\hspace{2mm}\textcolor{bluepoli}{|   }\hsp}{0pt}{\huge\bfseries \textcolor{bluepoli}
}

% Title format: section
\titleformat{\section}
{\color{bluepoli}\normalfont\Large\bfseries}
{\color{bluepoli}\thesection.}{1em}{}

% Title format: subsection
\titleformat{\subsection}
{\color{bluepoli}\normalfont\large\bfseries}
{\color{bluepoli}\thesubsection.}{1em}{}

% Title format: subsubsection
\titleformat{\subsubsection}
{\color{bluepoli}\normalfont\large\bfseries}
{\color{bluepoli}\thesubsubsection.}{1em}{}

% Shortening for setting no horizontal-spacing
\newcommand{\hsp}{\hspace{0pt}}

\makeatletter
% Renewcommand: cleardoublepage including the background pic
\renewcommand*\cleardoublepage{%
  \clearpage\if@twoside\ifodd\c@page\else
  \null
  \AddToShipoutPicture*{\BackgroundPic}
  \thispagestyle{empty}%
  \newpage
  \if@twocolumn\hbox{}\newpage\fi\fi\fi}
\makeatother

%For correctly numbering algorithms
\numberwithin{algorithm}{chapter}

%----------------------------------------------------------------------------
%	NEW COMMANDS DEFINED
%----------------------------------------------------------------------------

% EXAMPLES OF NEW COMMANDS
\newcommand{\bea}{\begin{eqnarray}} % Shortcut for equation arrays
\newcommand{\eea}{\end{eqnarray}}
\newcommand{\e}[1]{\times 10^{#1}}  % Powers of 10 notation

%----------------------------------------------------------------------------
%	ADD YOUR PACKAGES (be careful of package interaction)
%----------------------------------------------------------------------------

%----------------------------------------------------------------------------
%	ADD YOUR DEFINITIONS AND COMMANDS (be careful of existing commands)
%----------------------------------------------------------------------------

%----------------------------------------------------------------------------
%	BEGIN OF YOUR DOCUMENT
%----------------------------------------------------------------------------

\begin{document}

\fancypagestyle{plain}{%
\fancyhf{} % Clear all header and footer fields
\fancyhead[RO,RE]{\thepage} %RO=right odd, RE=right even
\renewcommand{\headrulewidth}{0pt}
\renewcommand{\footrulewidth}{0pt}}

%----------------------------------------------------------------------------
%	TITLE PAGE
%----------------------------------------------------------------------------

\pagestyle{empty} % No page numbers
\frontmatter % Use roman page numbering style (i, ii, iii, iv...) for the preamble pages

\puttitle{
	RASD document, % Title of the thesis
	name= Jacopo Piazzalunga, Mattia Piccinato, Gabriele Puglisi, % Author
	academicyear={2023-2024},  % Academic Year
} % These info will be put into your Title page 

%----------------------------------------------------------------------------
%	PREAMBLE PAGES: ABSTRACT (inglese e italiano), EXECUTIVE SUMMARY
%----------------------------------------------------------------------------
\startpreamble
\setcounter{page}{1} % Set page counter to 1

%----------------------------------------------------------------------------
%	LIST OF CONTENTS/FIGURES/TABLES/SYMBOLS
%----------------------------------------------------------------------------

% TABLE OF CONTENTS
\thispagestyle{empty}
\tableofcontents % Table of contents 
\thispagestyle{empty}
\cleardoublepage

%-------------------------------------------------------------------------
%	THESIS MAIN TEXT
%-------------------------------------------------------------------------
% In the main text of your thesis you can write the chapters in two different ways:
%
%(1) As presented in this template you can write:
%    \chapter{Title of the chapter}
%    *body of the chapter*
%
%(2) You can write your chapter in a separated .tex file and then include it in the main file with the following command:
%    \chapter{Title of the chapter}
%    \input{chapter_file.tex}
%
% Especially for long thesis, we recommend you the second option.

\addtocontents{toc}{\vspace{2em}} % Add a gap in the Contents, for aesthetics
\mainmatter % Begin numeric (1,2,3...) page numbering


% FIRST CHAPTER
% --------------------------------------------------------------------------
\chapter{Introduction}

\section{Purpose}

The purpose of the project CodeKataBattle (CKB) is to develop a platform where Students can practice coding together. Educators set up challenges (called Battles) within Tournaments, and Students work in teams to solve them. The platform checks their code automatically, giving scores based on how well it works, how quickly they finish, and how good their code quality is. Educators may optionally give extra scores by checking the work themselves, in a so-called Consolidation Stage which starts right after the end of every Battle. Students get ranked in Tournaments based on their scores obtained in teams, and can earn Badges for some achievements, in order to make learning programming more fun and competitive.

{\color{bluepoli}\rule{\linewidth}{0.1pt}}

\subsection{Goals}

{\color{bluepoli}\rule{\linewidth}{0.1pt}}

\begin{enumerate}
    \item[\textcolor{bluepoli}{G1}] Allows registered Students who enrolled according to the right modalities to participate in a Tournament of Code Kata Battles and take part in its Battles.
    \item[\textcolor{bluepoli}{G2}] Allows registered Educators to manage Tournaments for which they have been granted permission.
    \item[\textcolor{bluepoli}{G3}] Allows registered Students who participate in a Tournament of Code Kata Battles to be rewarded of different achievements.
    \item[\textcolor{bluepoli}{G4}] Allows registered Users to visualize information for which they have granted permission.
    \item[\textcolor{bluepoli}{G5}] Automates code evaluation process using GitHub Actions and some static analysis tools.
\end{enumerate}

{\color{bluepoli}\rule{\linewidth}{0.1pt}}

\section{Scope}

The main features which should be provided in order to achieve the aim of the project are:

\item \textcolor{bluepoli}{Creating Challenges:} Educators can make coding challenges (Battles) that Students can join, alone or in teams (groups), within the context of a Tournament.

\item \textcolor{bluepoli}{Using GitHub Actions for Code Submsissions:} Students are supposed to submit their code for a Battle in their GitHub repository, and the system must be informed of a new commit by a participating Student making use of GitHub Actions.

\item \textcolor{bluepoli}{Checking Code both Automatically and Manually:} The platform assigns a score to Students' code automatically, without the intervention of any Educator. Optionally, Educators may also decide to evaluate the code themselves.

\item \textcolor{bluepoli}{Rankings:} During each Battle, a live ranking of the involved teams is available, enabling participating Students to track their performance. Additionally, live Tournament rankings show how well each Student is performing in the Battles within a given Tournament.

\item \textcolor{bluepoli}{Badges for Achievements:} At the end of every Tournament, Students who achieved good results may be awarded with a special Badges, which are ruled by the Educator who created the Tournament.

The main goal is to help Students practice coding, giving them feedback and comparing their results to others.

\subsection{World phenomena}

{\color{bluepoli}\rule{\linewidth}{0.1pt}}

\begin{enumerate}
    \item[\textcolor{bluepoli}{WP1}] An Educator wants to create a Tournament competition.
    \item[\textcolor{bluepoli}{WP2}] A Student wants to participate in a Tournament competition.
    \item[\textcolor{bluepoli}{WP3}] A Student forks a Battle GitHub repo.
    \item[\textcolor{bluepoli}{WP4}] A Student sets up the update workflow using GitHub Actions.
    \item[\textcolor{bluepoli}{WP5}] An Educator wants to end a Tournament competition.
    \item[\textcolor{bluepoli}{WP6}] A User wants to visualize data about Tournaments, Battles, and Student profiles.
    \item[\textcolor{bluepoli}{WP7}] An Educator is given permission by another Educator to manage a Tournament.
\end{enumerate}

{\color{bluepoli}\rule{\linewidth}{0.1pt}}

\subsection{Shared phenomena}

{\color{bluepoli}\rule{\linewidth}{0.1pt}}

\subsubsection{Controlled by Machine}

\begin{enumerate}
    \item[\textcolor{bluepoli}{SP1}] The system notifies the Students about the creation of a Tournament.
    \item[\textcolor{bluepoli}{SP2}] The system notifies the groups about the creation of a Battle.
    \item[\textcolor{bluepoli}{SP3}] The system notifies the groups about the evaluation of a Battle.
    \item[\textcolor{bluepoli}{SP4}] The system notifies the Students about the end of a Tournament.s
    \item[\textcolor{bluepoli}{SP5}] The system shows some information about a Battle.
    \item[\textcolor{bluepoli}{SP6}] The system shows some information about a Tournament.
    \item[\textcolor{bluepoli}{SP7}] The system shows some information about a Student.
    \item[\textcolor{bluepoli}{SP8}] The system creates a GitHub repository containing the code kata.
    \item[\textcolor{bluepoli}{SP9}] The system assigns the score to the code of a group.
    \item[\textcolor{bluepoli}{SP10}] The system assigns a Badge to a Student.
\end{enumerate}

\subsubsection{Controlled by World}

\begin{enumerate}
    \item[\textcolor{bluepoli}{SP11}] An Educator creates a Tournament of code kata Battles.
    \item[\textcolor{bluepoli}{SP12}] An Educator defines Badges (see section 1.3) for the Tournament.
    \item[\textcolor{bluepoli}{SP13}] An Educator gives permission to another Educator to manage a Tournament.
    \item[\textcolor{bluepoli}{SP14}] An Educator creates a code kata Battle.
    \item[\textcolor{bluepoli}{SP15}] A Student creates a group for a Battle.
    \item[\textcolor{bluepoli}{SP16}] A Student invites another Student to join the group they created.
    \item[\textcolor{bluepoli}{SP17}] A Student joins a group for a Battle.
    \item[\textcolor{bluepoli}{SP18}] An API call is received by the system when a commit is pushed.
    \item[\textcolor{bluepoli}{SP19}] An Educator closes a Battle.
    \item[\textcolor{bluepoli}{SP20}] An Educator closes a Tournament.
    \item[\textcolor{bluepoli}{SP21}] An Educator evaluates the code of a Student.
\end{enumerate}

{\color{bluepoli}\rule{\linewidth}{0.1pt}}

\section{Definitions, Acronyms, Abbreviations}

\section{Revision history}

\section{Reference Documents}

\section{Document Structure}


% SECOND CHAPTER
% --------------------------------------------------------------------------
\chapter{Overall Description}

\section{Product perspective}

\section{Product functions}

\section{User characteristics}

\section{Assumptions, dependencies and constraints}


% THIRD CHAPTER
% --------------------------------------------------------------------------
\chapter{Specific requirements}

\section{External Interface Requirements}

\subsection{User Interfaces}

\subsection{Hardware Interfaces}

\subsection{Software Interfaces}

\subsection{Communication Interfaces}

\section{Functional Requirements}

\section{Performance Requirements}

\section{Design Constraints}

\subsection{Standards compliance}

\subsection{Hardware limitations}

\subsection{Any other constraint}

\section{Software System Attributes}

\subsection{Reliability}

\subsection{Availability}

\subsection{Security}

\subsection{Maintainability}

\subsection{Portability}

\chapter{Formal analysis using alloy}

\chapter{Effort spent}

\chapter{References}





\end{document}
